\documentclass{article}
\usepackage[utf8]{inputenc}
\usepackage[round]{natbib}

\title{Dissertation}
\author{Michael Gallagher}

\begin{document}

\maketitle

\tableofcontents

\section{Introduction}

\subsection{Background}

\subsubsection{Volatility}

Volatility in financial markets (huge discussion on volatility, why it happens, why prices doesn't behave around the mean as expected etc)...	

\subsubsection{Foreign Exchange}

In the past 30 years the foreign exchange markets has grown from an exclusive, locked down market to one of the most actively traded and liquid markets today. The market has many interesting properties compared to the more historically significant stock market, such as 24 hour trading, and the price being a good general reflection on the state of a country or unions economy. However, due to the recent surge in interest in this market it has not had the same level of study as the traditionally popular markets despite showing some interesting behaviours that may be of interest...

\subsubsection{Fuzzy Logic}

Some of the greatest difficulties with detecting certain patterns in technical analysis is determining whether a price is a valid part of a pattern when it is 'kind of' part of a pattern and 'kind of' not. Studies on technical analysis often define crisp ranges to group together prices, such as in \citep{foundations} where prices are seen to be equal in a pattern if they are within 1.5\% of their averages. This of course leaves us with the issue that prices within 1.49\% are observered as being part of a pattern but prices within 1.51\% are discarded in the same manner as a price that differs dramatically...

\subsection{Objectives}

\subsubsection{Benefits of Fuzzy Pattern Matching}

Fuzzy logic is still a new field of interest but with a wide range of applications. There is still much room for research that demonstrates the differences in performance between using a fuzzy logic system and using a traditional crisp logic system... (Need to actually do both then!)

\subsubsection{Applying Stock Market Analysis to Foreign Exchange Markets}

The methods described in this paper have been shown to be effective at predicting the price movement of companies on the Stock Exchange. There appears to be significant differences in the behaviour of the foreign exchange market and the Stock exchange, as told by ...

\subsubsection{Predictive Capability of Technical Analysis}

There is still much debate as to whether analysis of previous price action can be used to determine future price performance...

\subsubsection{Using Statistical Techniques to Analyse Price Movement}

I'm not sure where to throw this in, but it's important. This is essentially the only way the method my dissertation differs to previous work, so I need to justify it. (ie. I'm grouping the price using z-scores)...

\section{State of the art}

AIM: Write for someone not familiar with the subject

\subsection{Technical Analysis}

Technical Analysis is a form of financial analysis where past price behaviour is used to predict future price behaviour in financial markets. Proponents of this system claim that the movement of price, or change in some other factor such as volume, can be subject to recurring patterns, which can be used to predict future changes in price. Examples of this form of analysis include identifying trends in a graph of price movement, or concluding that a price is unlikely to go lower than the 52-week low.

\subsubsection{History}

The use of this form of analysis has been documented for several hundred years. One of the first uses of Technical Analysis dates back to Japan in the 18th Century AD. In 1710 Japan's rice market had matured enough to begin use rice coupons instead of trading physical rice. Their receipts were traded constituting the first Futures trades ever recorded \citep[p.15]{jcct1991}. 

Munehisa Homma was a rice merchant in Japan, who amassed great wealth trading these rice coupons and is sometimes credited as the Father of Japanese Candlestick Charting. Although it is unclear whether Homma actually used Japanese Candlestick Charting techniques, it is recorded that he used the history of price movement to gauge the emotions of the market at the time, ``When all are bearish, there is cause for prices to rise. When everyone is bullish, there is cause for the price to fall''. Following Homma different charting techniques were introduced, until the introduction and popularisation of Japanese Candlestick Charting for Technical Analysis in the mid 19th Century \citep[p.18]{jcct1994}.

In the west, Dow Theory was developed and expanded on in the early 20th Century, leading to works \citep{edwards2012technical} that directly influence Technical Analysis today.

Even with such a diverse history the use of Technical Analysis is still a hotly debated topic. Experts are divided in its use \citep{foundations}, the influential \textit{A Random Walk down Wall Street} concluding it is as valid as Alchemy when put under scientific scrutiny \cite[p.159]{randomwalk2012}. Nonetheless, it appears a significant number of analysts incorporate Technical Analysis, with a study by \cite{examininguse1997} finding a mean importance of 35\% was given to Technical Analysis by respondents in various investment banks, and \cite{cheung2000currency} characterising 30\% of Foreign-Exchange traders as Technical Traders. 

\subsubsection{Efficient Market Hypothesis}

One explanation for this debate is that Technical Analysis appears to contradict the once widely accepted Efficient Market Hypothesis. The Efficient Market Hypothesis claims that the price of a security is extremely efficient at reflecting the information about that security. When news becomes public the market will very efficiently arrive at a price for the security that reflects the impact of the news. This claim has major consequences, such that both Technical Analysis and Fundamental Analysis, which is analysing the health of a security by viewing information such as company earnings, are unable to help an investor achieve a greater profit than if a random selection of securities were invested in \citep{emhAndCritics}. 

While there is still a lot of support for this model, strict adherence to it does not offer the complete picture. If the markets were perfectly efficient then the price would simply jump between points as individual news was released. In reality there is a large amount of noise, volatility and uncertainty in the market. Very often it is not clear what impact some news should have. And of course the accuracy of market information and the availability of the information to the public add further uncertainty to the price movement.

The Random Walk Hypothesis says that even with these movements in price the market is still informationally efficient. It proposes that the price of a security represents a 'random walk' around its intrinsic value, and consequently that price movements can not be predicted. \cite{lo1988} reject this hypothesis however, but note that ``this rejection does not necessarily imply the inefficiency of stock-price formulation''.

One theory is that these uncertainties in price movement allow Technical Analysis to be utilised. \cite{indefenseof} show that past prices combined with other information, such as non-public information, can achieve unusual profit. They claim that non-price information creates opportunity that can be efficiently exploited using Technical Analysis. As non-public information is revealed to an investor, it is unclear as to the extent that this information has been revealed. If the investor knows the behaviour that a market would demonstrate if this information was widely distributed, he or she could examine the past price movements and determine a probability that the information has been revealed, and thus determine if there is still opportunity for a profitable investment.

While this demonstrates a use for past price data, it only demonstrates use when combined with other valuable information, leaving the usefulness of Technical Analysis as a general analysis tool an open question. More - wikipedia

\subsubsection{Academic Research}

So why is Technical Analysis still widely used if it contradicts popular financial theory? There are many possible explanations for this, as while there may be a lack of strong statistical evidence for its effectiveness, it is also far from debunked. 	

One of several explanations by \cite[p.45-71]{aronson2011evidence} is that humans suffer cognitive biases when determining the significance of visual patterns in their analysis. Many experts believe they can spot visual patterns such as price trends on a graph, and would take the claim that these trends are irrelevant as ridiculous. Technical Analysts take the significance of these patterns for granted. However when developing or practising a trading system based on patterns, they can succumb to confirmation bias by giving a higher significance to evidence that supports the pattern they're looking for as opposed to evidence for other patterns that might be available. Also when determining the efficacy of these systems based on past data, hindsight bias can lead to certain patterns appearing obvious after their formation. This leads an expert to think this pattern is easily detectable, even though while the pattern was forming the evidence for that particular pattern was indecisive, and possibly pointed to other, contradictory patterns \cite[p.62]{aronson2011evidence}.

Another explanation is that Technical Analysis appeals to behavioural psychology. If a large amount of investors are searching for similar patterns, then the outcome following these patterns may become a self-fulfilling prophecy. A study by \cite{examininguse1997} finds that 49\% of Foreign-Exchange Technical Analysts explicitly cite this as a reason. These points are disputed by Malkiel however, stating "The problem is that once such a regularity is known to market participants, people will act in a way that prevents it from happening in the future" \cite[p.162]{randomwalk2012}.

These explanations aren't really enough to justify such widespread acceptance, but they do demonstrate factors that might encourage someone to adopt a technical trading strategy. Fortunately with such a controversy surrounding technical analysis for several decades, academics have produced a significant amount of literature on the subject.

\cite{taprofitability} performed a survey of 95 modern studies of technical trading strategies. Among these, 56 found positive results, 20 obtained negative results and 19 indicated mixed results. The survey also found that different markets were better suited to Technical Analysis, such as foreign exchange.

But then the question still remains, if it is indeed effective, why is it so difficult to demonstrate this empirically?

Another explanation for this disagreement is the view that Technical Analysis lacks some of the scientific rigour applied to other techniques in common use. This is partly due to the difficulty in experimentally verifying some Technical Analysis techniques, many of which rely on visual pattern recognition in the eyes on an Expert. This subjectivity as to the existence of a pattern can sometimes lead to experts viewing differently or disagreeing about certain patterns, fuelling the controversy that Technical Analysis is nothing more than pseudo-analysis.

These difficulties in verifying different techniques scientifically have been countered in recent years with...

\subsection{Time-Series analysis}

Discussion of the various papers that this is inspired from. The TAIEX paper, the univeristy enrollment paper,  the various dual factor fuzzy logic time series papers.

\section{Method}

AIM: To leave no doubt in the reader that I have done this correctly, and covered all bases.

\subsection{Acquiring Data}

Data used was publicly available data obtained from [...]. The data was verified against several sources, and verified that data was complete. Data used was daily opening...

\subsection{Statistical Analysis of Data}

A clear picture was desired as to the significance of price movement. The use of z-scores in identifying price patterns was... (NB. This is different to previous papers that I am comparing to which makes comparisons difficult).

\subsection{Time Series Methodology}

How time series analysis works, taken largely from other papers...

\subsection{Crisp Logic System Implementation}

Essentially how I applied the time series analysis to what I have right now (using rolling window, using time series methodology). This section may not be necessary.

\subsection{Fuzzy Logic Inference System}

\subsubsection{Justification of Design}

Why I did mandami for example (nb. not yet decided).

\subsubsection{Implementation}

Step by step fuzzy logic system - fuzzification, inference, aggregation, defuzzification...

\section{Results \& Evaluation}
Stand back and evaluate what you have achieved and how well you have met the objectives. Evaluate your achievements against your objectives in section 1.2. Demonstrate that you have tackled the project in a professional manner. 

\subsection{RMSE Performance Indicator}

Outline how it has been used in other papers. Outline how we can draw direct comparisons...

\subsection{Results}

Results of the analysis of data. Show that the average price movement comes to approximately 0, as expected. Show the patterns discovered, show the significance of these using statistical analysis...

\subsubsection{Statistical significance of patterns}

Show the significance of certain patterns statistically using survey sampling techniques \& confidence interval...

\subsection{Forward testing}

See the error when testing against data that was not used to train the program...

\subsection{Comparisons}

Compare the results to previous papers.

\section{Conclusion}

Analysis of results against objectives...

\section{Future work}

What went well, what could be improved...

\bibliographystyle{plainnat}
\bibliography{dissertation}

\end{document}